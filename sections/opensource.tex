%-------------------------------------------------------------------------------
%	SECTION TITLE
%-------------------------------------------------------------------------------
\cvsection{Open Source}

%-------------------------------------------------------------------------------
%	CONTENT
%-------------------------------------------------------------------------------
\begin{cventries}

%---------------------------------------------------------
  \cventry
    {Simple alerting mechanism for apps running on Marathon} % Description
    {\href{https://github.com/ashwanthkumar/marathon-alerts}{marathon-alerts}} % Project title
    {March 2016} % Date(s)
    {} % Ignore
    {
    Marathon Alerts is a tool for monitoring the apps running on marathon. Inspired from tools like kubernetes-alerts and consul-alerts.
    This was initially built for Marathon 0.8.0, hence we don't use the event bus.
    }

%---------------------------------------------------------
  \cventry
    {CLI client to Mesosphere Marathon + WIP Package Manager} % Description
    {\href{https://github.com/ashwanthkumar/marathonctl}{marathonctl}} % Project title
    {Jan. 2016} % Date(s)
    {} % Ignore
    {
    CLI tool to access and deploy apps and services to \href{https://mesosphere.github.io/marathon/}{Marathon}.
    }

%---------------------------------------------------------
  \cventry
    {Scale applications on cloud} % Description
    {\href{https://github.com/ashwanthkumar/vamana2}{vamana2}} % Project title
    {Nov. 2015} % Date(s)
    {} % Ignore
    {
    Vamana2 helps you scale your applications on AWS (cloud agnostic actually) using custom application metrics.
    You can find my presentation of Vamana2 on Chennai October AWS Meetup \href{http://j.mp/to-vamana}{here}.
    }

%---------------------------------------------------------
  \cventry
    {Automatically migrate between cheapest AZs on AWS} % Description
    {\href{https://github.com/ashwanthkumar/matsya}{matsya}} % Project title
    {Oct. 2015} % Date(s)
    {} % Ignore
    {
    Matsya is a Java application that helps you move your fleet of machines across Availability Zones
    on AWS to be cost effective and fallback to On-Demand when there is a huge demand in Spot market.
    You can find my presentation on Matsya on Chennai DevOps Meetup \href{j.mp/to-matsya}{here}.
    }

%---------------------------------------------------------
  \cventry
    {Distributed Notification system built for Amazon Hackathon} % Description
    {\href{https://github.com/ashwanthkumar/meghaduta}{meghaduta}} % Project title
    {Sept. 2015} % Date(s)
    {} % Ignore
    {
    Meghadūta is a scalable (pun intended) notification system built at Amaz-ing Hackathon, Amazon Chennai.
    Check out the \href{https://github.com/ashwanthkumar/meghaduta/blob/master/README.md}{README.md} for design and problem statement.
    }

%---------------------------------------------------------
  \cventry
    {Wayback Clone} % Description
    {\href{https://github.com/ashwanthkumar/finder}{finder}} % Project title
    {Jan. 2015} % Date(s)
    {} % Ignore
    {
    Finder is a hobby project to build a Wayback clone without converting our existing crawled data which are in SequenceFiles to WARC / ARC formats.
    Architecture is very much inspired from Wayback but with a few changes. It can not only support HTML pages but finder can provide key+timestamp based access to any dataset.
    }

%---------------------------------------------------------
  \cventry
    {A Social Analytics System} % Description
    {\href{http://blog.ashwanthkumar.in/2012/05/blueignis-feature-screencast-closed.html}{blueignis}} % Project title
    {May 2012} % Date(s)
    {} % Ignore
    {
    It helps you identify what people talk about your product on the social media of the web.
    My Final year project built and scaled to handle ~1M messages per minute (on AWS). System was built on Storm and Hadoop.
    Find demo screencast of the product \href{http://blog.ashwanthkumar.in/2012/05/blueignis-feature-screencast-closed.html}{here}.
    }

%---------------------------------------------------------
  \cventry
    {Web Crawler app in PHP} % Description
    {\href{https://github.com/ashwanthkumar/scraphp}{scraphp}} % Project title
    {Jan. 2012} % Date(s)
    {} % Ignore
    {
    Scraphp (say Scraph, last p is silent) is a web crawling program, basically built to be a standalone executable which can be set a corn task to crawl websites and store extract useful content out of it.
    Initially created for hacker challenge posted by Indix on Jan 2012.
    }

%---------------------------------------------------------
\end{cventries}
