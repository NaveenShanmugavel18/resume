%-------------------------------------------------------------------------------
%	SECTION TITLE
%-------------------------------------------------------------------------------
\cvsection{Open Source}

%-------------------------------------------------------------------------------
%	CONTENT
%-------------------------------------------------------------------------------
\begin{cventries}

%---------------------------------------------------------
  \cventry
    {Toolkit to build distributed Data Systems} % Description
    {\href{https://github.com/ashwanthkumar/suuchi}{suuchi}} % Project title
    {Nov. 2016} % Date(s)
    {} % Ignore
    {
    Suuchi is toolkit to build distributed data systems, that uses gRPC under the hood as the communication medium. The overall goal of this project is to build pluggable components that can be easily composed by the developer to build a data system of desired characteristics.
    }

%---------------------------------------------------------
  \cventry
    {Simple alerting mechanism for apps running on Marathon} % Description
    {\href{https://github.com/ashwanthkumar/marathon-alerts}{marathon-alerts}} % Project title
    {Mar. 2016} % Date(s)
    {} % Ignore
    {
    Marathon Alerts is a tool for monitoring the apps running on marathon. Inspired from tools like kubernetes-alerts and consul-alerts.
    This was initially built for Marathon 0.8.0, hence we don't use the event bus.
    }

%---------------------------------------------------------
  \cventry
    {CLI client to Mesosphere Marathon + WIP Package Manager} % Description
    {\href{https://github.com/ashwanthkumar/marathonctl}{marathonctl}} % Project title
    {Jan. 2016} % Date(s)
    {} % Ignore
    {
    CLI tool to access and deploy apps and services to \href{https://mesosphere.github.io/marathon/}{Marathon}.
    }

%---------------------------------------------------------
  \cventry
    {Scale applications on cloud} % Description
    {\href{https://github.com/indix/vamana}{vamana}} % Project title
    {Nov. 2015} % Date(s)
    {} % Ignore
    {
    Vamana helps you scale your applications on AWS (cloud agnostic actually) using custom application metrics.
    You can find my presentation of Vamana on Chennai October AWS Meetup \href{http://j.mp/to-vamana}{here}.
    }

%---------------------------------------------------------
  \cventry
    {Automatically migrate between cheapest AZs on AWS} % Description
    {\href{https://github.com/indix/matsya}{matsya}} % Project title
    {Oct. 2015} % Date(s)
    {} % Ignore
    {
    Matsya is a Java application that helps you move your fleet of machines across Availability Zones
    on AWS to be cost effective and fallback to On-Demand when there is a huge demand in Spot market.
    You can find my presentation on Matsya on Chennai DevOps Meetup \href{j.mp/to-matsya}{here}.
    }

%---------------------------------------------------------
\end{cventries}
